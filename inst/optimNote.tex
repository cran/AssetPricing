\input{head2e}
\usepackage{amsmath,amssymb}
\newcommand{\vex}{\mbox{\boldmath $x$}}
\newcommand{\vv}{\mbox{\boldmath $v$}}
\begin{document}

I find that I keep feeling confused over the business of optimizing
with respect to price, particularly in the ``discrete price''
setting.  These notes are an attempt to set down some clarification
so as to prevent future confusion.

The situation is that we have a system of differential equations
for the expected values $v_q(t)$, explicitly
\begin{equation}
\dot{v}_q(t) = \lambda(t) (-v_q(t) + R_q(x,t \mid \vex))
\label{eq:dotv}
\end{equation}
where $R_q(x,t \mid \vex)$ is the expected revenue, under pricing
policy $\vex$, from a stock of size $q$ given an arrival at time $t$.
The expression for $R_q(x,t \mid \vex)$ is somewhat complicated and
differs between the singly indexed price setting and the doubly
indexed price setting.  The exact expression for $R_q(x,t \mid \vex)$
is however immaterial.  It depends (only) on the values of $x$,
$v_q(t), q = 1, \ldots, q_{\rm max}$ (and some parameters which are
assumed to be known).

To determine an optimal pricing policy we need to choose $x$ to
maximize $R_q(x,t \mid \vex)$.

The reason that I get confused is that it looks like $\dot{v}_q(t)$ (and
hence $v_q(t)$) depend on $R_q(x,t \mid \vex)$ which in turn depends on
$v_q(t)$ so that we have a circularity.  And so we're stuck.

But not so!  All we need is to be able to calculate $\dot{v}_q(t)$
as a function of $v_q(t)$ and $t$.  Then we have a well-defined
differential equation which we can solve numerically.  So we
simply need to be able to maximize $R_q(x,t \mid \vex)$ as a function
of a \emph{known} value of $v_q(t)$.  Then we can calculate the right
hand side of (\ref{eq:dotv}) for any value of $v_q(t)$ and $t$, and
that's all we need in order to be able to solve (\ref{eq:dotv})
numerically.

Well, O.K.  I'm playing just a \emph{little} bit fast and loose
here to simplify the exposition.  I said that $R_q(x,t \mid \vex)$
depends on $v_q(t), q = 1, \ldots, q_{\rm max}$, not just on the
$v_q(t)$ that we are currently looking at.  But it's really alright.
The easiest way to see this is to think in terms of a \emph{vector}
system of differential equations
\begin{equation}
\dot{\vv}(t) = {\cal F}(\vv(t),t,\boldsymbol{R}^*(t))
\label{eq:vecdotv}
\end{equation}
where $\vv(t) = \tr{(v_1(t), \ldots, v_{q_{\rm max}}(t))}$,
and each entry of $\boldsymbol{R}^*(t)$ is the maximum over $x$
of the corresponding entry of the vector of expected revenues
$\boldsymbol{R}(x,t)$. This latter quantity depends (only) on $x$
and $\vv(t)$.

We can (in the discrete price setting) calculate the right hand
side of (\ref{eq:vecdotv}) for any value of $\vv(t)$ and whence
(\ref{eq:vecdotv}) is a well-defined (vector) differential equaion,
which we can solve.  And Bob's your uncle.  Good old Uncle Bob.

\end{document}
